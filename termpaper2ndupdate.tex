\documentclass[12pt,letterpaper, onecolumn]{exam}
\usepackage{amsmath}
\usepackage{amssymb}
\usepackage[lmargin=71pt, tmargin=1.2in]{geometry}  %For centering solution box
\usepackage{graphicx}
\lhead{Term Paper\\}
\rhead{Image Recognition\\}
% \chead{\hline} % Un-comment to draw line below header
\thispagestyle{empty}   %For removing header/footer from page 1
\begin{document}

\begingroup  
    \centering
    \LARGE TERM PAPER\\
    \LARGE DATABASE MANAGEMENT SYSTEMS\\[0.5em]
    \large \today\\[0.5em]
    \large INTERACTIVE DATA EXPLORATION AND VISUALISATION\par
    \large Roll Number - 19111026\par
    \large BME/ 6th sem / B.Tech\par
\endgroup
\rule{\textwidth}{0.4pt}
\pointsdroppedatright   %Self-explanatory
\printanswers
\renewcommand{\solutiontitle}{\noindent\textbf{Ans:}\enspace}   %Replace "Ans:" with starting keyword in solution box


    \section{Introduction}
    A database management system (or DBMS) is essentially nothing more than a computerized data-keeping system. Users of the system are given facilities to perform several kinds of operations on such a system for either manipulation of the data in the database or the management of the database structure itself. Database Management Systems (DBMSs) are categorized according to their data structures or types.
    \\\\
    Mainframe sites tend to use a hierarchical model when the data structure (not data values) of the data needed for an application is relatively static. For example, a Bill of Material (BOM) database structure always has a high level assembly part number, and several levels of components with subcomponents. The structure usually has a component forecast, cost, and pricing data, and so on. The structure of the data for a BOM application rarely changes, and new data elements (not values) are rarely identified. An application normally starts at the top with the assembly part number, and goes down to the detail components.
    \\\\
    Hierarchical and relational database systems have common benefits. RDBMS has the additional, significant advantage over the hierarchical DB of being non-navigational. By navigational, we mean that in a hierarchical database, the application programmer must know the structure of the database. The program must contain specific logic to navigate from the root segment to the desired child segments containing the desired attributes or elements. The program must still access the intervening segments, even though they are not needed.
    
    \section{What is Interactive Data Visualisation ?}
    Interactive data visualization refers to the use of modern data analysis software that enables users to directly manipulate and explore graphical representations of data. Data visualization uses visual aids to help analysts efficiently and effectively understand the significance of data. Interactive data visualization software improves upon this concept by incorporating interaction tools that facilitate the modification of the parameters of a data visualization, enabling the user to see more detail, create new insights, generate compelling questions, and capture the full value of the data.
    
    \section{2 Challenges and Opportunities}
    Designing a system for interactive data exploration with a human-in-the-loop front-end requires solving a set of very unique research challenges while also opening the door to several interesting opportunities. In this section, we first outline some of the requirements and challenges, followed by an overview of some of the unique opportunities to address them.
    
    

\end{document}
